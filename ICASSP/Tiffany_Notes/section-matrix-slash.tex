\documentclass[11pt,letter]{article}
\usepackage[utf8]{inputenc}
\usepackage{lipsum}
\usepackage{amsfonts}
\usepackage{amsmath}
\usepackage{bm}
\usepackage{graphicx}
\usepackage{algorithmic}
\usepackage{algorithm2e}
\usepackage[margin=3cm]{geometry}
\usepackage[colorinlistoftodos]{todonotes}


      
\begin{document}
\subsection{Matrix Adapted Orthogonal Polynomials Method}
In order to approximate the matrix function $f(\mathbf{A})\mathbf{b}$, we estimate the probability distribution of the eigenvalues of $\mathbf{A}$ as a discrete measure $d\lambda_N$, generate a set of orthogonal polynomials $\{\pi_k\}$ with respect to $d\lambda_N$, and represent $f$ as a linear combination of $\{\pi_k\}$. Then, $f(\mathbf{A})\mathbf{b}$ can be approximated with a truncated series of polynomial expansion.\\

First, we use the Kernel Polynomial Method (KPM) to estimate the cumulative distribution $F$ of the eigenvalues of $\mathbf{A}$, and convert it to a discrete measure. We take a set of $N-2$ evenly spaced points with the two endpoints as the initial grid on the interval $[\lambda_{\min},\lambda_{\max}]$, where $\lambda_{\min}$ and $\lambda_{\max}$ are the smallest and largest eigenvalues of $\mathbf{A}$. Assume $\lambda_{\min}=t_1<t_2<\cdots<t_{N-1}<t_N=\lambda_{\max}$, we generate the weights for the grid points $\{t_i\}$ by taking the difference $F(t_i+\Delta t/2)-F(t_i-\Delta t/2)$ for $i=2,3,\cdots,N-1$, where $\Delta t= (\lambda_{\max}-\lambda_{\min})/(N-2)$. The weights at both endpoints are set to 1. Then, we normalize all the weights by dividing with the total number of eigenvalues.\\

Next, we generate a set of orthogonal polynomials $\{\pi_k\}$ with respect to the discrete measure with the method described in section \ref{ssec:op_discrete_measure}.\\

The function $f$ can be approximated with its projection onto the subspace spanned by $\{\pi_0,\pi_1,\cdots,\pi_k\}$, i.e. the first $K+1$ terms in its polynomial expansion. We first take the inner product of $f$ and $\pi_k$ on the discrete measure $d\lambda_N$:
\begin{equation}
\label{eqn:inner_product}
\left<f,\pi_k\right>_{d\lambda_N}=\sum_{i=1}^N w_if(t_i)\pi_k(t_i),\,\,\,k=0,1,2,\cdots.
\end{equation}
% (2.2.1), P90, Gautschi
For each grid point $t_i$, we have
\begin{equation}
\label{eqn:expansion_gti}
f(t_i)=\sum_{k=0}^{N} \gamma_k\pi_k(t_i)\approx \sum_{k=0}^K \gamma_k\pi_k(t_i),\,\,\,i=1,2,\cdots,N,
\end{equation}
where $$\gamma_k=\frac{\left<f,\pi_k\right>_{d\lambda_N}}{\left<\pi_k,\pi_k\right>_{d\lambda_N}}.$$
\noindent For an arbitrary matrix $\mathbf{A}\in\mathbb{C}^{N\times N}$ and vector $\mathbf{b}\in\mathbb{C}^N$, it follows from the three-term recurrence relationship in \eqref{eqn:three_term_recur} that
\begin{equation}
\label{eqn:three_term_recur_piab}
\begin{split}
\pi_{k+1}(\mathbf{A})\mathbf{b}=(\mathbf{A}-\alpha_k \mathbf{I})\pi_k(\mathbf{A})\mathbf{b}-&\beta_k \pi_{k-1}(\mathbf{A})\mathbf{b},\,\,\, k=0,1,2,\cdots,\\
\pi_{-1}(\mathbf{A})\mathbf{b}=\bm{0},&\,\,\,\pi_{0}(A)\mathbf{b}=\mathbf{b}.
\end{split}
\end{equation}

\noindent Therefore, the matrix function $f(\mathbf{A})\mathbf{b}$ can be approximated with a finite linear combination of $\{\pi_k(\mathbf{A})\mathbf{b}\}$, with $k\in\{0,\cdots,K\}$, as follows:

\begin{equation}
\label{eqn:expansion_fab}
f(\mathbf{A})\mathbf{b}\approx p(\mathbf{A})\mathbf{b}=\sum_{k=0}^{N-1} \gamma_k\pi_k(\mathbf{A})\mathbf{b}\approx \sum_{k=0}^K \gamma_k\pi_k(\mathbf{A})\mathbf{b}.
\end{equation}

\end{document}
