\documentclass[11pt,letter]{article}
\usepackage[utf8]{inputenc}
\usepackage{lipsum}
\usepackage{amsfonts}
\usepackage{amsmath}
\usepackage{bm}
\usepackage{graphicx}
\usepackage{algorithmic}
\usepackage{algorithm2e}
\usepackage[margin=3cm]{geometry}
\usepackage[colorinlistoftodos]{todonotes}


      
\begin{document}
\section{Orthogonal Polynomials with respect to a Discrete Measure}

\subsection{A Discrete Measure}

A discrete $N$-point measure $d\lambda_N$ supported on a finite number of distinct points can be represented with the set of $N$ support points $\{t_i\}$ and the corresponding weights $\{w_i\}$, $i=1,2,\cdots,N$. Given functions $u$ and $v$, their inner product on $d\lambda_N$ is defined as:
\begin{equation}
\label{eqn:def_inner_product}
\left<u,v\right>_{d\lambda_N}=\sum_{i=1}^N w_iu(t_i)v(t_i).
\end{equation}
% (1.1.12), P4, Gautschi
If $\left<u,v\right>_{d\lambda_N}=0$, $u$ and $v$ are orthogonal with respect to ${d\lambda_N}$.
% (1.1.2), P1, Gautschi
The norm of $u$ is defined as:
$$||u||_{d\lambda_N}=\sqrt[]{\left<u,u\right>_{d\lambda_N}}=\left(\sum_{i=1}^N w_iu(t_i)^2\right)^{1/2}.$$

\subsection{Objection}
Given the set of $N$ support points $\{t_i\}$ with positive weights $\{w_i\}$, our goal is to establish a set of monic polynomials $\{\pi_k\}$, $k=0,1,2,\cdots,N-1$, that are mutually orthogonal with respect to the discrete measure $d\lambda_N$.

\subsection{Orthogonal Polynomials with respect to $d\lambda_N$}
The monic orthogonal polynomials with respect to a discrete measure $d\lambda_N$, denoted as $\pi_k(\cdot)=\pi_k(\cdot,d\lambda_N),k=0,1,2,\cdots,N$, satisfy the three-term recurrence relationship:
\begin{equation}
\label{eqn:three_term_recur}
\begin{split}
\pi_{k+1}(t)=(t-\alpha_k)\pi_k(t)-&\beta_k\pi_{k-1}(t),\,\,\, k=0,1,2,\cdots,N-2,\\
\pi_{-1}(t)=0,&\,\,\,\pi_{0}(t)=1,
\end{split}
\end{equation}
where the recurrence coefficients $\{\alpha_k\}$ and $\{\beta_k\}$ are determined by
\begin{equation}
\label{eqn:recur_coef}
\begin{split}
\alpha_k&=\frac{\left<t\pi_k,\pi_k\right>_{d\lambda_N}}{\left<\pi_k,\pi_k\right>_{d\lambda_N}},\,\,\, k=0,1,2,\cdots,N-1,\\
\beta_k&=\frac{\left<\pi_k,\pi_k\right>_{d\lambda_N}}{\left<\pi_{k-1},\pi_{k-1}\right>_{d\lambda_N}},\,\,\, k=1,2,\cdots,N-1,\\
&\beta_0=\left<\pi_0,\pi_0\right>_{d\lambda_N}=\left<1,1\right>_{d\lambda_N}.
\end{split}
\end{equation}
% Theorem 1.27, P10, Gautschi; discrete version and uniform convergence on P91

If we normalize every monic polynomial $\pi_k$ by $\tilde{\pi}_k=\pi_k/||\pi_k||_{d\lambda_N}$, we obtain a set of orthonormal polynomials $\{\tilde{\pi}_k\}$ with a similar three-term recurrence relationship:
\begin{equation}
\label{eqn:three_term_recur_normal}
\begin{split}
\sqrt[]{\beta_{k+1}}\tilde{\pi}_{k+1}(t)=(t-\alpha_k)\tilde{\pi}_k(t)&-\sqrt[]{\beta_k}\tilde{\pi}_{k-1}(t),\,\,\, k=0,1,2,\cdots,N-2,\\
\tilde{\pi}_{-1}(t)=0,\,\,\,&\tilde{\pi}_{0}(t)=1/\sqrt[]{\beta_0}.
\end{split}
\end{equation}

For $n\leq N$, the $n^{th}$ order Jacobi matrix associated with the measure $d\lambda_N$ is a symmetric tridiagonal matrix defined in the form:
\begin{equation}
\label{eqn:def_jn}
\mathbf{J_n}=\mathbf{J_n}(d\lambda_N)=\begin{bmatrix}\alpha_0 & \sqrt[]{\beta_1} & & 0\\
\sqrt[]{\beta_1} & \alpha_1 & \ddots & \\
& \ddots & \ddots & \sqrt[]{\beta_{n-1}} \\
0& & \sqrt[]{\beta_{n-1}} & \alpha_{n-1}
\end{bmatrix},
\end{equation}
where $\{\alpha_k\}$ and $\{\beta_k\}$ are from \eqref{eqn:recur_coef}. We write $\mathbf{J_n}(d\lambda_N)$ if we want to exhibit the measure $d\lambda_N$.\\

%Consider $n<N$ and 
Let $\bm{\tilde{\pi}}(t)=(\tilde{\pi}_0(t),\tilde{\pi}_1(t),\cdots,\tilde{\pi}_{n-1}(t))^T$, the three-term recurrence relationship in \eqref{eqn:three_term_recur_normal} can be written in the matrix form:
\begin{equation}
\label{eqn:three_term_recur_matrix}
t\bm{\tilde{\pi}}(t)=\mathbf{J_n}\bm{\tilde{\pi}}(t)+\sqrt[]{\beta_n}\tilde{\pi}_n(t)\mathbf{e_n},
\end{equation}
where $\mathbf{e_n}=(0,0,\cdots,1)^T$ is the last column of the identity matrix $I_n$.\\
% 1.3.18, G, p13, Thm 1.31

Denote the zeros of $\tilde{\pi}_n$ (i.e. zeros of $\pi_n$) as $\{\tau_v\}$, $v=1,2,\cdots,n$. Setting $t=\tau_v$ in \eqref{eqn:three_term_recur_matrix}, we obtain that $\{\tau_v\}$ are eigenvalues of $\mathbf{J_n}$, and $\bm{\tilde{\pi}}(\tau_v)$ are the corresponding eigenvectors. Thus, we can decompose the Jacobi matrix in the following way:
\begin{equation}
\label{eqn:decomposition_of_jn}
\mathbf{J_n}=\mathbf{VD_\tau V^T},
\end{equation}
where $\mathbf{V}=(\bm{\tilde{\pi}}(\tau_1),\bm{\tilde{\pi}}(\tau_2),\cdots,\bm{\tilde{\pi}}(\tau_n))$ and $\mathbf{D_\tau}=diag(\tau_1,\tau_2,\cdots,\tau_n)$.\\

% Since $\mathbf{J_n}$ is symmetric, columns of $\mathbf{V}$ are mutually orthogonal. Consider
% $$\mathbf{V}^T\mathbf{V}=\mathbf{D_{\pi}},\,\,\,\mathbf{D_{\pi}}=diag(d_0,d_1,\cdots,d_{n-1}),$$ where the $v^{th}$ diagonal element of $\mathbf{D_{\pi}}$ is $$d_{v-1}=\sum_{k=0}^{n-1}(\tilde{\pi}_k(\tau_v))^2.$$\\



It can be shown that there exists an orthogonal similarity transformation from $\{\tau_v\}$ in $\mathbf{D_\tau}$ to the recurrence coefficients $\{\alpha_k\}$ and $\{\beta_k\}$ in $\mathbf{J_n}$:
\begin{equation}
\label{eqn:magic_identity}
\begin{bmatrix}1 &\mathbf{0}^T\\ \mathbf{0} & \mathbf{V}\end{bmatrix} \begin{bmatrix}1 &\sqrt[]{\bm{\lambda}}^T\\\sqrt[]{\bm{\lambda}} &\mathbf{D_\tau} \end{bmatrix} 
\begin{bmatrix}1 &\mathbf{0}^T\\ \mathbf{0} & \mathbf{V}^T\end{bmatrix} = \begin{bmatrix}1 & \sqrt[]{\beta_{0}}\mathbf{e_1}^T\\ \sqrt[]{\beta_{0}}\mathbf{e_1} &\mathbf{J_n} \end{bmatrix},
\end{equation}
where $\mathbf{J_n}=\mathbf{J_n}(d\lambda_N)$ is from \eqref{eqn:def_jn}, $\mathbf{V}$, $\mathbf{D_\tau}$ are from \eqref{eqn:decomposition_of_jn}, and 
$$\sqrt[]{\bm{\lambda}}=(\sqrt{\lambda_1},\sqrt{\lambda_2},\cdots,\sqrt{\lambda_n})^T,$$
 with $\lambda_v$ denoting the weight of the point $\tau_v$ in the $n$-point Gauss quadrature rule.\\
% 3.1.9, P154, Proof of identity P152-155, Gautschi


\subsection{A Stable Method to Compute Recurrence Coefficients}
In order to evaluate the discrete orthogonal polynomials $\{\pi_k\}$, we have to compute the recurrence coefficients $\{\alpha_k\}$ and $\{\beta_k\}$. An intuitive method, known as the Stieltjes precedure, focuses on the three-term recurrence relationship in \eqref{eqn:three_term_recur} and computes $\{\pi_k\}$ and $\{\alpha_k\}$, $\{\beta_k\}$ alternatively. However, this procedure is numerically unstable as $k$ increases. Below we introduce a stable method to obtain $\{\alpha_k\}$ and $\{\beta_k\}$ through the Lanczos process.

\subsubsection{The Lanczos Process}
The $k^{th}$ Krylov subspace of a matrix $\mathbf{A}\in\mathbb{C}^{N\times N}$ and a nonzero vector $\mathbf{b}\in\mathbb{C}^N$ is defined as $$K_k(\mathbf{A},\mathbf{b})=span\{\mathbf{b},\mathbf{A}\mathbf{b},\cdots,\mathbf{A}^{k-1}\mathbf{b}\},\,\,\, 1\leq k\leq N.$$
Given a Hermitian matrix $\mathbf{A}\in\mathbb{C}^{N\times N}$ and a vector $\mathbf{b}\in\mathbb{C}^N$, the Lanczos process computes the Hessenberg reduction of $\mathbf{A}$, defined as $\mathbf{H}=\mathbf{Q}^{\star}\mathbf{A}\mathbf{Q}$, where $\mathbf{H}\in\mathbb{C}^{N\times N}$ is symmetric tridiagonal, and $\mathbf{Q}\in\mathbb{C}^{N\times N}$ is unitary. The columns of $\mathbf{Q}$, denoted as \{$\mathbf{q_1},\mathbf{q_2},\cdots,\mathbf{q_N}$\}, where $\mathbf{q_1}=\mathbf{b}/||\mathbf{b}||_2$, form a orthonormal basis of $K_N(\mathbf{A},\mathbf{b})$.\\

% In order to approximate $f(\mathbf{A})\mathbf{b}$ to a given order $k\,(1\leq k \leq N)$, we project $A$ and $b$ onto the $k^{th}$ Krylov subspace, operate with $f$ and expand the result back to $K_N(\mathbf{A},\mathbf{b})$. Denote the first $k$ columns of $\mathbf{Q}$ as $\mathbf{Q_k}\in\mathbb{C}^{N\times k}$, and let $\mathbf{H_k}=\mathbf{Q_k}^{\star}\mathbf{AQ_k}\in\mathbb{C}^{k\times k}$, $$f(\mathbf{A})\mathbf{b}=\mathbf{Q}^{\star}f(\mathbf{H})\mathbf{Q}\mathbf{b}\approx \mathbf{Q_k}^{\star}f(\mathbf{H_k})\mathbf{Q_k}\mathbf{b}=||\mathbf{b}||_2\mathbf{Q_k}^{\star}f(\mathbf{H_k})\mathbf{Q_k}\mathbf{e_1}.$$


\subsubsection{The Stable Method (?)}
Starting with $N$ support points $\{t_i\}$ and their weights $\{w_i\}$, we define vector $\sqrt[]{\mathbf{w}}$ and diagonal matrix $\mathbf{D_t}$ as follows: $$\sqrt[]{\mathbf{w}}=(\sqrt{w_1},\sqrt{w_2},\cdots,\sqrt{w_N})^T,$$ $$\mathbf{D_t}=diag(t_1,t_2,\cdots,t_N).$$
%P98, Gautschi
Similar to \eqref{eqn:magic_identity}, we have
\begin{equation}
\label{eqn:magic_identity_original}
\begin{bmatrix}1 &\mathbf{0}^T\\ \mathbf{0} & \mathbf{Q_1}^T\end{bmatrix} \begin{bmatrix}1 &\sqrt[]{\mathbf{w}}^T\\\sqrt[]{\mathbf{w}} &\mathbf{D_t} \end{bmatrix} 
\begin{bmatrix}1 &\mathbf{0}^T\\ \mathbf{0} & \mathbf{Q_1}\end{bmatrix} = \begin{bmatrix}1 & \sqrt[]{\beta_{0}}\mathbf{e_1}^T\\ \sqrt[]{\beta_{0}}\mathbf{e_1} &\mathbf{J_N} \end{bmatrix}.
\end{equation}

Accordingly, we construct a Hermitian matrix $\mathbf{A}=\begin{bmatrix}1 &\sqrt[]{\mathbf{w}}^T\\\sqrt[]{\mathbf{w}} &\mathbf{D_t} \end{bmatrix}$ and take $\mathbf{b}=\mathbf{e_1}$. The Lanczos process leads to a symmetric tridiagonal matrix as on the right of \eqref{eqn:magic_identity_original}, the major and minor diagonal elements of which are the recurrence coefficients of interest. \\

For any given order $K\leq N-1$, , we tridiagonalize $\mathbf{A}$ iteratively till the result contains $\mathbf{J_K}$. With the recurrence coefficients $\{\alpha_k\}$ and $\{\beta_k\}$ from $\mathbf{J_K}$, and the three-term recurrence relationship in \eqref{eqn:three_term_recur}, we can evaluate $\{\pi_k\},k=0,1,\cdots,K$ at all support points $\{t_i\},i=1,2,\cdots,N$.\\





\end{document}
